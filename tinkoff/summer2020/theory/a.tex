\documentclass[12pt,a4paper]{report}

\usepackage{cmap}
\usepackage[T2A]{fontenc}
\usepackage[utf8]{inputenc}
\usepackage[russian]{babel}
\usepackage{graphicx}
\usepackage{amsthm,amsmath,amssymb}
\usepackage{listings}
\usepackage{color}
\usepackage{xcolor}
\usepackage{array}
\usepackage{epigraph}
\usepackage{import}
\usepackage{pdfpages}
\usepackage{transparent}

\newcommand{\incfig}[2][1]{%
    \def\svgwidth{#1\columnwidth}
    \import{./figures/}{#2.pdf_tex}
}

\pdfsuppresswarningpagegroup=1



\usepackage[russian,colorlinks=true,urlcolor=red,linkcolor=blue]{hyperref}
\usepackage{enumerate}
\usepackage{datetime}
\usepackage{fancyhdr}
\usepackage{lastpage}
\usepackage{verbatim}
\usepackage{tikz}
\usetikzlibrary{arrows,decorations.markings,decorations.pathmorphing}
\usepackage{pgfplots}

\usepackage{ifthen}
\usepackage{mathtools}
\usepackage{multirow}
\usepackage{paralist}
\usepackage{tcolorbox}
\usepackage{todonotes}

\tcbuselibrary{breakable}

\graphicspath{{pictures/}}
\DeclareGraphicsExtensions{.pdf,.png,.jpg}
%\usepackage{tabls}
%\usepackage{tabularx}
%\usepackage{xifthen}
%\listfiles
\newtheorem{theorem}{Теорема}

\newtheorem*{lemma}{Лемма}
\newtheorem*{remark}{Замечение}
\newtheorem{problem}[theorem]{Задача}
\sloppy
\newenvironment{temp}{\begin{tcolorbox}[
    arc=0mm,
    colback=white,
    colframe=red!60!black,
    title=temp,
    fonttitle=\sffamily,
    breakable
]}{\end{tcolorbox}}
\newcommand*{\defeq}{\stackrel{\text{def}}{=}}
\setlength\epigraphwidth{.8\textwidth}

\newlength{\tmplen}
\newlength{\tmpwidth}
\newcounter{listcounter}

\begin{document}
Для начала поймем, когда нельзя построить перестановку по данным ограничениям. Для этого, построим граф, где вершинами будут индексы элементов перестановки, а ребра построим по слежующему правилу: для каждого ограничения из каждого элемента из отрезка $[a_i;b_i]$ проведем ориентированное ребро в каждый элемент из отрезка $[c_i;d_i]$. У нас получился полный граф сравнений. Заметим, что если в данном графе есть цикл, то получается, что существует какой-то элемент $x$ который меньше самого себя. Тогда по данным ограничениям построить перестановку невозможно. \par
Иначе, так как наш граф не имеет циклов, то мы можем построить топологическую сортировку и расставить числа в порядке, обратном топологической сортировке. Данное решение работает за $\mathcal{O}(n+L^2)$ \par
Теперь для каждого индекса перестановки будем поддерживать количество еще необработанных ограничений, где данный индекс лежит в отрезке $[c_i;d_i]$. Также будем поддерживать очередь из индексов, в которых данная величина(для удобства назовем её $cnt_i$) достигла $0$. Также для каждого ограничения будем поддерживать величину $cntseg_i$ --- количество элементов из отрезка $[a_i;b_i]$, которые еще не обработаны. Теперь рассмотрим очередной элемент $x$ из очереди. Поставим на позицию $x$ минимальное непоставленное число. Теперь рассмотрим все ограничения, где $x$ лежит в отрезке $[a_i;b_i]$. У каждого такого отрезка уменьшим $cntseg_i$ на 1, и, если $cntseg_i$ стало равно $0$, то у каждого элемента $y$ из отрезка $[c_i;d_i]$ вычтем $1$ из $cnt_y$. Если $cnt_y$ стало равно $0$, то добавим $y$ в очередь. \par
Заметим, что если после конца работы алгоритма остались необработанные элементы, то в графе сравнений есть цикл. Также несложно заметить, что мы рассматриваем элементы в порядке, обратном топологичской сортировке. \par
Теперь оценим асимптотику: каждый элемент мы добавим и удалим из очереди ровно 1 раз. Для каждого элемента мы обновляем $cntseg$ за количество отрезков, в которых он находится в отрезке $[a_i;b_i]$, т.е. суммарно мы обновим $cntseg$ не больше, чем $\mathcal{O}(L)$ раз. Также $cnt$ мы обновим не больше, чем $\mathcal{O}(L)$ раз из тех же соображений. Следовательно алгоритм будет работать за $\mathcal(O)(n + L)$. 
\end{document}
